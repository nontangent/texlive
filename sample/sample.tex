\documentclass[twocolumn, a4j]{article}
\usepackage{multirow}
\usepackage{amsmath,amssymb}
\usepackage[T1]{fontenc}
% \usepackage[dvipdfmx]{graphicx}

\title{サンプル用のTexファイル}
\renewcommand{\thefootnote}{\fnsymbol{footnote}}
\author{Nozomu Miyamoto\footnotemark[2] nontan@sfc.wide.ad.jp}
\renewcommand{\thefootnote}{\arabic{footnote}}
\date{\today}

\begin{document}

\twocolumn[
\begin{@twocolumnfalse}
  \maketitle
  \vspace{-6mm}
  \begin{abstract}
    ここに概要を書きましょう.
  \end{abstract}
  \vspace{2mm}
\end{@twocolumnfalse}
]

\renewcommand{\thefootnote}{\fnsymbol{footnote}}
\footnotetext[2]{慶應義塾環境情報学部 NECO Lab.}
\renewcommand{\thefootnote}{\arabic{footnote}}

\section{はじめに}

   hoge hoge hoge hoge

\renewcommand{\refname}{参考文献}
\begin{thebibliography}{数字}
  \bibitem[Shigeo 2000]{Shigeo 2000} Shigeo Mastubara and Makoto Yoko, Fraud-Free Exchange Mechanisms in Electronic Commerce, Journal of Japanese Society for Artificial Intelligence(2000).
\end{thebibliography}

\end{document}
